%% Preamble. Used by calendar and by cover
\pdfminorversion=3 %% This is required by lulu.com: http://connect.lulu.com/t5/Interior-Formatting/PDF-Creation-Settings-How-can-I-be-sure-my-PDF-will-print/ta-p/33681
%\RequirePackage{pdf13}
\usepackage[T2A]{fontenc}
\usepackage[english,russian]{babel}
\usepackage[utf8]{inputenc}
\usepackage{cmap}
\usepackage{hyperxmp}
\usepackage{hyperref}
\usepackage{ifthen}
\usepackage{color}
\usepackage{graphicx}
\usepackage{tikz}
\usepackage{import}
\usepackage{titling}
\usepackage{wallpaper}
\usetikzlibrary{calendar,fit}
\ifdefined\PreambleQuick
\else %% These packages slow down the compilation
 \usepackage{auto-pst-pdf}
 \usepackage{pst-barcode}
\fi
\usepackage{ccicons}
\usepackage{expl3,xparse}

%\pdfminorversion=3 %% This is required by lulu.com: http://connect.lulu.com/t5/Interior-Formatting/PDF-Creation-Settings-How-can-I-be-sure-my-PDF-will-print/ta-p/33681
\pdfobjcompresslevel=3
\pdfcompresslevel=9

%% Customization of the template
%% MUST be included before \hypersetup!!! Author is taken from customization
%% Put your customization here
%% Defines file names with the photos for each month
%% (Located in the folder "photos"

%% Prefix for the web site where calendar data (photos etc) is located.
%% This site can be generated from the "web" folder
\newcommand{\webprefix}{http://vrepin.org/cal2014/}

% Author
\author{Vitaly Repin}

\newcommand{\Jan}{espoo}
\newcommand{\JanDscRU}{Парк Нууксио, Финляндия}
\newcommand{\JanDscEN}{Nuuksio National Park, Finland}

\newcommand{\Feb}{riga}
\newcommand{\FebDscRU}{Рига, Латвия}
\newcommand{\FebDscEN}{Riga, Latvia}

\newcommand{\Mar}{savonlinna}
\newcommand{\MarDscRU}{Олафсборг, Финляндия}
\newcommand{\MarDscEN}{Olavinlinna, Finland}

\newcommand{\Apr}{ruskeala}
\newcommand{\AprDscRU}{Горный парк «Рускеала», Россия}
\newcommand{\AprDscEN}{Ruskeala park, Russia}

\newcommand{\May}{hoboi}
\newcommand{\MayDscRU}{Озеро Байкал, Россия}
\newcommand{\MayDscEN}{Lake Baikal, Russia}

\newcommand{\Jun}{kapchuk}
\newcommand{\JunDscRU}{Плато Путорана, Россия}
\newcommand{\JunDscEN}{Putorana plateau, Russia}

\newcommand{\Jul}{peterhof}
\newcommand{\JulDscRU}{Петергоф, Россия}
\newcommand{\JulDscEN}{Peterhof, Russia}

\newcommand{\Aug}{piter}
\newcommand{\AugDscRU}{Летний сад, Россия}
\newcommand{\AugDscEN}{Summer Garden, Russia}

\newcommand{\Sep}{porkkala}
\newcommand{\SepDscRU}{Порккала, Финляндия}
\newcommand{\SepDscEN}{Porkkala, Finland}

\newcommand{\Oct}{irkutsk}
\newcommand{\OctDscRU}{Иркутская область, Россия}
\newcommand{\OctDscEN}{Irkutsk region, Russia}

\newcommand{\Nov}{zunmurino}
\newcommand{\NovDscRU}{Тункинская долина, Россия}
\newcommand{\NovDscEN}{Tunkin valley, Russia}

\newcommand{\Dec}{nuuksio}
\newcommand{\DecDscRU}{Эспоо, Финляндия}
\newcommand{\DecDscEN}{Espoo, Finland}

%% Names on the title and back covers in russian and english
\newcommand{\caltitleRU}{Времена года}
\newcommand{\caltitleEN}{The Four Seasons}

%% Holidays which shall be marked in red
\newcommand{\markRed}{}
\newcommand{\restoreMarkRed}{%
\renewcommand{\markRed}{if (Sunday,
      Saturday,
      equals=01-01,
      equals=01-02,
      equals=01-03,
      equals=01-04,
      equals=01-06,
      equals=01-07,
      equals=01-08,
      equals=03-10,
      equals=05-01,
      equals=05-02,
      equals=05-09,
      equals=06-12,
      equals=06-13,
      equals=11-03,
      equals=11-04,
      equals=02-23) [red]
}}


%% All the days which shall be marked by any sings, notes etc
\newcommand{\markDays}{
if (equals=2014-04-14,
      equals=2014-01-06,
      equals=2014-01-13) [mark-week day]
  if (Monday,
      equals=01-01,
      equals=02-01,
      equals=03-01,
      equals=04-01,
      equals=05-01,
      equals=06-01,
      equals=07-01,
      equals=08-01,
      equals=09-01,
      equals=10-01,
      equals=11-01,
      equals=12-01) [week number]
  \markRed
  if (equals=01-01,
      equals=01-07,
      equals=02-23,
      equals=03-08,
      equals=05-01,
      equals=06-12,
      equals=11-04,
      equals=04-06,
      equals=04-12,
      equals=06-22,
      equals=07-13,
      equals=07-20,
      equals=08-17,
      equals=10-05,
      equals=09-01,
      equals=09-28,
      equals=10-26,
      equals=12-12
      ) [flag-flying day]
  if (equals=10-04) [flag-space day]
  if (equals=07-27) [flag-navy day]
  if (equals=12-20) [flag-kgb day]
  if (equals=08-02) [flag-airborne day]
  if (equals=09-14) [flag-military day]
  if (equals=05-09) [flag-victory day]
  if (equals=01-27,
      equals=09-08,
      equals=05-27) [flag-spb day]
  if (equals=04-20,
      equals=01-06,
      equals=01-14,
      equals=01-18,
      equals=01-19,
      equals=03-02,
      equals=03-03,
      equals=02-15,
      equals=02-22,
      equals=03-15,
      equals=03-22,
      equals=03-29,
      equals=04-13,
      equals=03-09,
      equals=05-29,
      equals=06-07,
      equals=06-08,
      equals=07-07,
      equals=08-14,
      equals=08-19,
      equals=08-28,
      equals=08-29,
      equals=10-14,
      equals=07-12,
      equals=04-07,
      equals=04-29,
      equals=06-09,
      equals=09-27,
      equals=09-11,
      equals=09-21,
      equals=11-01,
      equals=12-04,
      between=04-14 and 04-20 %% Страстная неделя
  ) [flag-orthodox day]
  if (equals=01-01) [observance=Новый год]
  if (equals=01-07) [observance=Рождество\\ Христово]
  if (equals=02-23) [observance=День защитника Отечества]
  if (equals=03-08) [observance=Международный женский день]
  if (equals=04-06) [observance=День геолога]
  if (equals=05-01) [observance=Праздник Весны и Труда]
  if (equals=05-09) [observance=День Победы]
  if (equals=11-04) [observance=День народного единства]
  if (equals=06-12) [observance=День России]
  if (equals=04-12) [observance=День\\ космонавтики]
  if (equals=06-22) [observance=День памяти и скорби]
  if (equals=07-20) [observance=День\\ металлурга]
  if (equals=08-02) [observance=День ВДВ\\ Ильин день]
  if (equals=08-17) [observance=День\\ Воздушного\\ Флота России]
  if (equals=07-13) [observance=День рыбака]
  if (equals=07-27) [observance=День ВМФ]
  if (equals=09-01) [observance=День знаний]
  if (equals=09-11) [observance=Иван Головосек]
  if (equals=09-14) [observance=День танкиста]
  if (equals=09-28) [observance=День машино\-строителя]
  if (equals=10-04) [observance=День Космических войск]
  if (equals=10-05) [observance=День учителя]
  if (equals=10-26) [observance=День\\ автомобилиста]
  if (equals=12-12) [observance=День\\ Конституции]
  if (equals=12-20) [observance=День чекиста]
  %% St. Petersburg (Leningrad) days
  if (equals=01-27) [observance=Снятие блокады Ленинграда]
  if (equals=09-08) [observance=Начало блокады Ленинграда]
  if (equals=05-27) [observance=Основание Санкт-Петербурга]
  %% Orthodoxy dates
  if (equals=01-14) [observance=Обрезание\\ Господне]
  if (equals=02-22,
  	equals=03-15,
	equals=03-22,
	equals=03-29) [observance=Родительская суббота]
  if (equals=01-19) [observance=Крещение\\ Господне]
  if (equals=01-06) [observance=Рождественский\\ сочельник]
  if (equals=01-18) [observance=Крещенский\\ сочельник]
  if (equals=03-02) [observance=Прощеное\\ воскресенье]
  if (equals=03-03) [observance=Чистый\\ понедельник]
  if (equals=02-15) [observance=Сретение\\ Господне]
  if (equals=02-24) [observance=Встреча\\ Масленицы]
  if (equals=04-07) [observance=Благовещение]
  if (equals=04-13) [observance=Вербное\\ воскресенье]
  if (equals=04-17) [observance=Великий\\ Четверток]
  if (equals=04-18) [observance=Великий\\ Пяток]
  if (equals=04-16) [observance=Великая\\ Среда]
  if (equals=04-15) [observance=Великий\\ Вторник]
  if (equals=04-19) [observance=Великая\\ Суббота]
  if (equals=04-14) [observance=Великий\\ Понедельник]
  if (equals=04-20) [observance=Пасха]
  if (equals=04-29) [observance=Радоница]
  if (equals=03-09) [observance=Торжество\\ Православия]
  if (equals=05-29) [observance=Вознесение\\ Господне]
  if (equals=06-07) [observance=Троицкая\\ родительская суббота]
  if (equals=06-08) [observance=Троица]
  if (equals=06-09) [observance=День Святого Духа]
  if (equals=08-14) [observance=Медовый Спас]
  if (equals=08-19) [observance=Преображение\\ Господне]
  if (equals=08-28) [observance=Успение\\ Пресвятой\\ Богородицы]
  if (equals=08-29) [observance=Третий Спас]
  if (equals=09-21) [observance=Рождество\\ Пресвятой\\ Богородицы]
  if (equals=09-27) [observance=Воздвижение Креста\\ Господня]
  if (equals=10-14) [observance=Покров\\ Пресвятой\\ Богородицы]
  if (equals=11-01) [observance=Димитриевская родительская суббота]
  if (equals=07-07) [observance=Рождество Иоанна\\ Предтечи]
  if (equals=07-12) [observance=День Петра и Павла]
  if (equals=12-04) [observance=Введение\\ Богородицы во храм]
}


\hypersetup{%
pdftitle={\thetitle},
pdfauthor={\theauthor},
pdfcopyright={This work is licensed under a Creative Commons Attribution-ShareAlike 3.0 Unported License},
pdfsubject={Calendar},
pdfkeywords={\Keywords},
pdflicenseurl={http://creativecommons.org/licenses/by-sa/3.0/},
pdfcaptionwriter={\theauthor},
pdfcontactcity={\ContactCity},
pdfcontactcountry={\ContactCountry},
pdfcontactemail={\ContactEmail},
pdflang={en}
}

\restoreMarkRed

%% Adapted from pgf source
\def\pgfcalendarmonthname#1{%
  \translate{\ifcase#1\or {Январь January}\or Февраль February\or Март March\or Апрель April\or
    Май\or Июнь\or Июль\or Август\or Сентябрь\or Октябрь\or
    Ноябрь\or Декабрь\fi}%
}


% Adapted from http://www.texample.net/tikz/examples/changing-the-default-calendar-layout/
\newcommand{\genYearCal}[1]{
\begin{tikzpicture}[every calendar/.style={
            month label above centered,
            month text={\textsc{\small\%mt}},
            every day/.append style={font=\tiny\itshape},
            if={(weekend) [red]},
            week list,
        }]
	\matrix[column sep=1em, row sep=3em, ampersand replacement=\&] { #1 };
\end{tikzpicture}}

%% Boxes with small calendars are NOT included if \PreambleQuick is defined (to boost the compile time)
\ifdefined\PreambleQuick
  \message{\string\PreambleQuick\ is defined. NOT including calendar-related definitions / box creations}
\else
  \message{\string\PreambleQuick\ is NOT defined. Including calendar-related definitions / box creations}

  \newsavebox{\yearCalFirst}
  \savebox{\yearCalFirst}{\genYearCal{
            \calendar[dates=2015-01-01 to 2015-01-last]; \&
            \calendar[dates=2015-02-01 to 2015-02-last]; \\
            \calendar[dates=2015-03-01 to 2015-03-last]; \&
            \calendar[dates=2015-04-01 to 2015-04-last]; \\
  }}

  %% Generates small calendar
  %% Argument 1: Month
  %% Argument 2: Year
  %% Argument 3: Month name in russian
  \newcommand{\genSmallCal}[3]{
  \begin{tabular}{c}
  \small #3 #2\\
  \begin{tikzpicture}[every day/.style={font=\small\selectfont}]]
   \calendar[dates=#2-#1-01 to #2-#1-last,
            week list,
	    day xshift=1cm,
	    day yshift=.65cm]
   \markRed
  ;
  \end{tikzpicture}
  \end{tabular}
  }

  \newsavebox{\smallCalDec}
  \newsavebox{\smallCalFeb}
  \newsavebox{\smallCalDecO}
  \newsavebox{\smallCalJan}
  \newsavebox{\smallCalMar}
  \newsavebox{\smallCalApr}
  \newsavebox{\smallCalMay}
  \newsavebox{\smallCalJun}
  \newsavebox{\smallCalJul}
  \newsavebox{\smallCalAug}
  \newsavebox{\smallCalSep}
  \newsavebox{\smallCalOct}
  \newsavebox{\smallCalNov}
  \newsavebox{\smallCalJanN}


  %% Small calendars for the months: Dec, 2013 - Jan 2015
  \savebox{\smallCalDecO}{\genSmallCal{12}{2013}{ДЕКАБРЬ}}
  \savebox{\smallCalFeb}{\genSmallCal{02}{2014}{ФЕВРАЛЬ}}
  \savebox{\smallCalJan}{\genSmallCal{01}{2014}{ЯНВАРЬ}}
  \savebox{\smallCalMar}{\genSmallCal{03}{2014}{МАРТ}}
  \savebox{\smallCalApr}{\genSmallCal{04}{2014}{АПРЕЛЬ}}
  \savebox{\smallCalMay}{\genSmallCal{05}{2014}{МАЙ}}
  \savebox{\smallCalJun}{\genSmallCal{06}{2014}{ИЮНЬ}}
  \savebox{\smallCalJul}{\genSmallCal{07}{2014}{ИЮЛЬ}}
  \savebox{\smallCalAug}{\genSmallCal{08}{2014}{АВГУСТ}}
  \savebox{\smallCalSep}{\genSmallCal{09}{2014}{СЕНТЯБРЬ}}
  \savebox{\smallCalOct}{\genSmallCal{10}{2014}{ОКТЯБРЬ}}
  \savebox{\smallCalNov}{\genSmallCal{11}{2014}{НОЯБРЬ}}
  \savebox{\smallCalDec}{\genSmallCal{12}{2014}{ДЕКАБРЬ}}
  %% 2015: We don't know the holidays yet. Marking only weekends with red
  \renewcommand{\markRed}{if (weekend) [red]}
  \savebox{\smallCalJanN}{\genSmallCal{01}{2015}{ЯНВАРЬ}}
  \restoreMarkRed

%% Adapted from http://tex.stackexchange.com/a/10199/4771
\makeatletter%
\tikzoption{day headings}{\tikzstyle{day heading}=[#1]}
\tikzstyle{day heading}=[]
\tikzstyle{day letter headings}=[
    execute before day scope={ \ifdate{day of month=1}{%
      \pgfmathsetlength{\pgf@ya}{\tikz@lib@cal@yshift}%
      \pgfmathsetlength\pgf@xa{\tikz@lib@cal@xshift}%
      \pgftransformyshift{-\pgf@ya}
      \pgftransformyshift{-.45cm}
      \foreach \d/\l in {0/ПОНЕДЕЛЬНИК MONDAY,1/ВТОРНИК TUESDAY,2/СРЕДА WEDNESDAY,3/ЧЕТВЕРГ THURSDAY,%
                         4/ПЯТНИЦА FRIDAY,5/СУББОТА SATURDAY,6/ВОСКРЕСЕНЬЕ SUNDAY} {
        \pgf@xa=\d\pgf@xa%
        \pgftransformxshift{\pgf@xa-\cellwidth/2}%
        \pgftransformyshift{\pgf@ya}%
	\ifnum\ifnum\d=5 1\else\ifnum\d=6 1\else0\fi\fi
   	=1 %
	\node[above=-0.5ex,day heading,align=center,text width=3cm,color=red,font=\tiny]{\l};
	\else
	\node[above=-0.5ex,day heading,align=center,text width=3cm,font=\tiny]{\l};
 	\fi
      }
    }{}%
  }%
]
\makeatother%
%% End

%% Adapted from pgf source
\def\pgfcalendarmonthname#1{%
  \translate{\ifcase#1\or ЯНВАРЬ JANUARY\or ФЕВРАЛЬ FEBRUARY\or МАРТ MARCH\or АПРЕЛЬ APRIL\or
    МАЙ MAY\or ИЮНЬ JUNE\or ИЮЛЬ JULY\or АВГУСТ AUGUST\or СЕНТЯБРЬ SEPTEMBER\or ОКТЯБРЬ OCTOBER\or
    НОЯБРЬ NOVEMBER\or ДЕКАБРЬ DECEMBER\fi}%
}


\makeatletter
\tikzstyle{month label above centered}=[%
  execute before day scope={%
    \ifdate{day of month=1}{%
      {
        \pgfmathsetlength{\pgf@xa}{\tikz@lib@cal@xshift}%
        \pgf@xb=\tikz@lib@cal@width\pgf@xa%
        \advance\pgf@xb by-\pgf@xa%
        \pgf@xb=.5\pgf@xb%
        \pgftransformxshift{\pgf@xb}%
        \pgftransformxshift{-\cellwidth/2}%
        \pgfmathsetlength{\pgf@y}{\tikz@lib@cal@yshift}%
        \pgftransformyshift{0.333\pgf@y}
        \tikzmonthcode%
      }
    }{}},
  every month/.append style={anchor=base}
]
\makeatother
%% End

\ExplSyntaxOn

%% Adapted from http://tex.stackexchange.com/a/56214/4771
%%
% first of all we define the user level commands
\NewDocumentCommand{\addtext}{ m }{ \bdaycal_input_add:n { #1 } }
\NewDocumentCommand{\addtextyear}{ mm }{ \bdaycal_input_add_y:nn { #1 } { #2 } }
\NewDocumentCommand{\showtext}{ }{ \bdaycal_output_direct: }

% allocate variable:
% a sequence for global storage of the inputs;
\seq_new:N \g_bdaycal_input_seq

% store globally an input in the sequence
\cs_new:Npn \bdaycal_input_add:n #1
 {
  \seq_gput_right:Nn \g_bdaycal_input_seq { #1 }
 }

\cs_new:Npn \bdaycal_input_add_y:nn #1 #2
 {
  \seq_gput_right:Nn \g_bdaycal_input_seq { #1 ~ ( \int_to_arabic:n
    { \pgfcalendarifdateyear - #2 } ) }
 }

% how to output in direct order; we simply do a mapping function calling
% \bdaycal_print:n after incrementing the counter
\cs_new_protected:Npn \bdaycal_output_direct:
 {
  \seq_map_inline:Nn \g_bdaycal_input_seq
   {
    \bdaycal_print:n { ##1 }
   }
  \seq_gclear:N \g_bdaycal_input_seq
 }

% the printing macro; change here for adapting to your wishes
\cs_new:Npn \bdaycal_print:n #1
 {
  #1 \par
 }
%% End

%% http://www.tondering.dk/claus/cal/week.php#calcweekno
\int_new:N \l_week_number_year_int
\int_new:N \l_week_number_month_int
\int_new:N \l_week_number_day_int
\int_new:N \l_week_number_a_int
\int_new:N \l_week_number_b_int
\int_new:N \l_week_number_c_int
\int_new:N \l_week_number_s_int
\int_new:N \l_week_number_e_int
\int_new:N \l_week_number_f_int
\int_new:N \l_week_number_g_int
\int_new:N \l_week_number_d_int
\int_new:N \l_week_number_n_int
\int_new:N \l_week_number_W_int

\cs_new:Nn \week_number:nnn {

  \int_set:Nn \l_week_number_year_int { #1 }
  \int_set:Nn \l_week_number_month_int { #2 }
  \int_set:Nn \l_week_number_day_int { #3 }

  \int_compare:nNnTF { \l_week_number_month_int } < { 3 } % jan or feb
  { % true

    \int_set:Nn \l_week_number_a_int { \l_week_number_year_int - 1 }

    \int_set:Nn \l_week_number_b_int {
      \int_div_truncate:nn { \l_week_number_a_int } { 4 }
      - \int_div_truncate:nn { \l_week_number_a_int } { 100 }
      + \int_div_truncate:nn { \l_week_number_a_int } { 400 }
    }

    \int_set:Nn \l_week_number_c_int {
      \int_div_truncate:nn { \l_week_number_a_int - 1 } { 4 }
      - \int_div_truncate:nn { \l_week_number_a_int - 1 } { 100 }
      + \int_div_truncate:nn { \l_week_number_a_int - 1 } { 400 }
    }

    \int_set:Nn \l_week_number_s_int {
      \l_week_number_b_int - \l_week_number_c_int }

    \int_zero:N \l_week_number_e_int

    \int_set:Nn \l_week_number_f_int { \l_week_number_day_int - 1
      + 31 * ( \l_week_number_month_int - 1 ) }

  } % end true
  { % false

    \int_set_eq:NN \l_week_number_a_int \l_week_number_year_int

    \int_set:Nn \l_week_number_b_int {
      \int_div_truncate:nn { \l_week_number_a_int } { 4 }
      - \int_div_truncate:nn { \l_week_number_a_int } { 100 }
      + \int_div_truncate:nn { \l_week_number_a_int } { 400 }
    }

    \int_set:Nn \l_week_number_c_int {
      \int_div_truncate:nn { \l_week_number_a_int - 1 } { 4 }
      - \int_div_truncate:nn { \l_week_number_a_int - 1 } { 100 }
      + \int_div_truncate:nn { \l_week_number_a_int - 1 } { 400 }
    }

    \int_set:Nn \l_week_number_s_int {
      \l_week_number_b_int - \l_week_number_c_int }

    \int_set:Nn \l_week_number_e_int { \l_week_number_s_int + 1 }

    \int_set:Nn \l_week_number_f_int { \l_week_number_day_int
      + \int_div_truncate:nn {
        153 * ( \l_week_number_month_int - 3 ) + 2 } { 5 }
      + 58 + \l_week_number_s_int }

  } % end false

  \int_set:Nn \l_week_number_g_int {
    \int_mod:nn { \l_week_number_a_int + \l_week_number_b_int } { 7 }  }

  \int_set:Nn \l_week_number_d_int {
    \int_mod:nn { \l_week_number_f_int + \l_week_number_g_int
      - \l_week_number_e_int } { 7 }  }

  \int_set:Nn \l_week_number_n_int {
    \l_week_number_f_int + 3 - \l_week_number_d_int }

  \int_compare:nNnTF { \l_week_number_n_int } < { 0 }
  { %true

    \int_set:Nn \l_week_number_W_int { 53
      - \int_div_truncate:nn { \l_week_number_g_int
        - \l_week_number_s_int } { 5 } }

  } % end true
  { % false

    \int_compare:nNnTF { \l_week_number_n_int } > { 364
      + \l_week_number_s_int }
    { % true

      \int_set:Nn \l_week_number_W_int { 1 }

    } % end true
    { % false

      \int_set:Nn \l_week_number_W_int { \int_div_truncate:nn {
          \l_week_number_n_int } { 7 } + 1 }

    } % end false

  } % end false

}

\pgfkeys{/tikz/week~number/.code =
  {
    \week_number:nnn { 
      \pgfcalendarifdateyear } { 
      \pgfcalendarifdatemonth } { 
      \pgfcalendarifdateday }
    \addtext{\textit{\int_to_arabic:n { \l_week_number_W_int }} }
  }
}

\ExplSyntaxOff


\pgfkeys{/tikz/observance/.code =
  {
    \addtext{#1}
  }
}

\newcommand{\TikzDayCode}{}
\newcommand{\restoreTikzDayCode}{\renewcommand{\TikzDayCode}{\node (lower right) at (0,0) [above left,font=\Huge] {\tikzdaytext};}}
\restoreTikzDayCode

\pgfkeys{/tikz/day code =
  {
    \TikzDayCode
    \node (upper left) at (-\cellwidth,\cellheight)
    [below right,align=left,text width=\cellwidth-\pgflinewidth,
    font=\tiny,black] {\showtext};
    \node (lower left) at (-\cellwidth,0) {};
    \node[rounded corners, draw,
          fit=(lower right) (upper left) (lower left),
          inner sep=1mm] {};
  }
}

\pgfkeys{/tikz/inner sep = 0pt}

\pgfkeys{/tikz/day xshift=\cellwidth+3.5mm}

\pgfkeys{/tikz/day yshift=\cellheight+2mm+2mm}

\newlength{\cellheight}
\setlength{\cellheight}{20mm}
\newlength{\cellwidth}
\setlength{\cellwidth}{26.85mm}

%% Background for selected (emphasized) weeks
\definecolor{selectedbg}{HTML}{EEEEEE}

\pgfkeys{/tikz/mark-week day/.code =
  {
     \filldraw[selectedbg,rounded corners=2mm] (-\cellwidth-3mm,-2mm) rectangle (7*\cellwidth-3.5mm,\cellheight+2mm);
  }
}

%% Argument 1: Start date
%% Argument 2: End data
\newcommand{\calmonthRaw}[2]{%
\vspace*{1cm}%
\centerline{%
\begin{tikzpicture}[thick]
  \calendar[dates=#1 to #2,
            week list,
            month label above centered,
            month text={\%mt \%y0},
            day headings={font=\footnotesize\scshape},
            day letter headings]
  %% Defined in customization.tex. The core of the calendar
  \markRed
  \markDays
  ;
\end{tikzpicture}}\bigskip}

%% Returns the file name of the photo for the month
%% Argument: month number: 1..12
\newcommand{\GetMonthImage}[1]{%
\ifnum#1=1\Jan%
\fi%
\ifnum#1=2\Feb%
\fi%
\ifnum#1=3\Mar%
\fi%
\ifnum#1=4\Apr%
\fi%
\ifnum#1=5\May%
\fi%
\ifnum#1=6\Jun%
\fi%
\ifnum#1=7\Jul%
\fi%
\ifnum#1=8\Aug%
\fi%
\ifnum#1=9\Sep%
\fi%
\ifnum#1=10\Oct%
\fi%
\ifnum#1=11\Nov%
\fi%
\ifnum#1=12\Dec%
\fi%
}

%% Returns the URL with the description of the image corresponding to the month
%% Argument: month number: 1..12
\newcommand{\GetMonthURL}[1]{%
\webprefix\ifnum#1=1january
\fi%
\ifnum#1=2february%
\fi%
\ifnum#1=3march%
\fi%
\ifnum#1=4april%
\fi%
\ifnum#1=5may%
\fi%
\ifnum#1=6june%
\fi%
\ifnum#1=7july%
\fi%
\ifnum#1=8august%
\fi%
\ifnum#1=9september%
\fi%
\ifnum#1=10october%
\fi%
\ifnum#1=11november%
\fi%
\ifnum#1=12december%
\fi%
}


%% Argument: month
\newcommand{\calmonth}[1]{\calmonthRaw{2014-#1-01}{2014-#1-last}}

%% Creates Bar Code with the URL
\newcommand{\descurl}[1]{%
\begin{pspicture}(1in,.1in)
\psbarcode{#1}{eclevel=H}{qrcode}
\end{pspicture}
}

%% Generates "small" calendar (2 such calendars are in every month's page) 
\newcommand{\smallmonths}[3]{\usebox{#2}\hfill\raisebox{-2cm}{\descurl{\GetMonthURL{#1}}}\hfill\usebox{#3}\newpage}

\subimport*{../../resources/}{flags}

\fi %% If \PreambleQuick is undefined!!!

\pagestyle{empty}

